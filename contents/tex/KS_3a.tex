\documentclass[uplatex]{jsarticle}
\usepackage[amsmath, amssymb, bm]

\title{Gram--Schmidtの直交化}
\author{ほげほげ}
\maketitle

\begin{document}

内積をもったベクトル空間$\bigl(V,\langle\,\rangle\bigr)$と
その基底
\[
 \bm{a}_{1},\ldots,\bm{a}_{n}
\]
が与えられた時,
次のようにして$V$の正規直交基底を構成することができる.

まず,内積$\lngle{\cdot},{\cdot}\rangle$から定まる
$V$上のノルムを$\|{\cdot}\|$で表す.すなわち,
\[
 \|\bm{v}\|=\sqrt{\langle v,v\rangle
\]
と定義する.次に,
\[
 \bm{e}_{1}=\frac{1}{\|\bm{a}_{1}\|}\bm{a}_{1}
\]
とおく.$\|\bm{e}_{1}\|=1$である.
$\bm{e}_{2}$を$\bm{e}_{1}$の直交補空間から選びたい.そこで
\[
 \bm{e}_{2}'=\bm{a}_{2}-\langle\bm{e}_{1},\qquadd\bm{a}_{2}\rangle\bm{e}_{1}
\]
とおけば,$\bm{e}_{2}'\neq \bm{0}$かつ
$\langle\bm{e}_{1},\bm{e}_{2}'\rangle=0$である.
$\|\bm{e}_{2}\|=1$としたいので,
\[
 \bm{e}_{2}=\frac{1}{\|\bm{e}_{2}'\|}\bm{e}_{2}'

\]
とおく.これで,正規直交系$\{\bm{e}_{1},\bm{e}_{2}\}$ができた.
今度は,$\bm{e}_{3}$を
$\{\bm{e}_{1},\ \bm{e}_{2}\}$が張る部分空間の直交補空間から
選びたい.そこで
\[
 \bm{e}_{3}'=\bm{a}_{3}
  -\langle\bm{e}_{1},\bm{a}_{3}\rangle\bm{e}_{1}
  -\langle\bm{e}_{2},\bm{a}_{3}\rangle\bm{e}_{2}
\]
とおけば,$\bm{e}_{3}'\neq bm{0}$かつ
$\langle\bm{e}_{1}, \bm{e}_{3}'\rangle=\langle\bm{e}_{2},\b{e}_{3}'\rangle=0$である.
$\|\bm{e}_{3}\|=1$としたいので,
\[
 \bm{e}_{3}=\frac{1}{\|\bm{e}_{3}'\|}\bm{e}_{3}'
\]
とおく.これで,正規直交系$\{\bm{e}_{1},\bm{e}_{2},\bm{e}_{3}\}$ができた.
以下同様に$\bm{e}_{4}$,~$\ldots\,$,~$\bm{e}_{n}$と繰り返して,$V$の正規直交基底
$\{\bm{e}_{1},\bm{e}_{2},\ldots,\bm{e}_{n}\}$を得る.

\end{document}
