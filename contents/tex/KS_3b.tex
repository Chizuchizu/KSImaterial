\documentclass{jsarticle}
\usepackage{amsmath, amssymb}

\title{極限と積分の順序交換}
\begin{document}
\author{自分の名前に書き換える}
\maketitle

一般に,関数の極限と積分を交換することはできない.そのような例を以下に示す.

$n=1,~2,~3.~\ldots$に対して,$f_{n}:\mathbb{R}\to\mathbb{R}$を
\begin{align*}
f_{n}(x)=
  \begin{cases}
  \frac{1}{n}x~(0\leq x\leq n)\\
  2-\frac{1}{n}x~(n\leq x\leq 2n)\\
  0~(\text{otherwise})
  \end{cases}
\end{align*}
と定める.このとき,
\begin{align}
 \lim_{n\to\infty}\int_{0}^{\infty}f_{n}(x)dx
 \neq\int_{0}^{\infty}\lim_{n\to\infty}f_{n}(x)dx\label{infty}
\end{align}
であることを確認しよう.まず,左辺の値を計算する.任意の$n$に対して,\\
$\int_{0}^{\infty}f_{n}(x)dx=1$\\
であるから,
\[
 \lim_{n\to\infty}\int_{0}^{\infty}f_{n}(x)dx=1
\]
がわかった.次に,右辺の値を計算する.$f_{n}$の定義より,
\[
 \lim{}_{n\to\infty}f_{n}(x)\equiv 0
\]
だから,
\[
 \int_{0}^{\infty}\lim_{n\to\infty}f_{n}(x)dx=0
\]
がしたがう。これで,(\ref{infty})が正しいことが確かめられた.\\
このような現象は,積分区間が有限でないから起きたのだろうか.
実は,そうではない.次のような例もある.

$n=1,~2,~3, ...$に対して,$g_{n}:(0,1]\mapsto\mathbb{R}$を
\begin{align*}
g_{n}(x)=
  \begin{cases}
  2n-2n^{2}x & (0<x≦\frac{1}{n})\\
  0 & (otherwise)
  \end{cases}
\end{align*}
と定める.このとき,
\begin{align}
 \lim_{n\to\infty}\int_{0}^{1}g_{n}(x)\mathrm{dx}
 \neq\int_{0}^{l}\lim_{n\to\infty}g_{n}(x)dx\label{fin}
\end{align}
であることを確認しよう.まず,左辺の値を計算する.任意の$n$に対して、
\[
 \int_{0}^{1}g_{n}(x)dx=2
\]
であるから,
\[
 \lim_n\to\infty\int_{0}^{1}h_{n}(x)dx=1
\]
がわかった.次に,右辺の値を計算する.$g_{n}の定義より,$
\[
 \lim_{n\to\infty}g_{n}(x)\cong 0
\]
だから,
\[
 \int_{0}^{\infty}\lim_{n\to\infty}g_{n}(x)dx=0
\]
がしたがう.これで,[\ref{fin}]が正しいことが確かめられた.
\end{document}
